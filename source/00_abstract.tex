Most studies on human cognitive behavior are performed in controlled laboratory settings, which can limit validity and the understanding of cognition in more naturalistic settings. The study of eye-movements in everyday tasks has revealed that attention is largely goal-directed and allocated based on task-requirements. The aim of the present study is to further investigate task-oriented eye movements during interaction with novel objects in more natural settings. 
In this study, we investigated the gaze dynamics during interaction with three-dimensional tool models in a virtual environment in two different experiments.. We introduce a 2x2x2 factorial design with factors task, familiarity and orientation of the tool. Participants were instructed to either lift or use (factor task) a given tool which was either familiar or unfamiliar (factor familiarity) to them and which was presented in one of two different orientations where the tool handle was presented either on the left or right side (factor orientation). The two experiments varied in the the naturalness of interactions with the tool models and the immersion of the environment.
We used linear mixed models to model the odds of fixations on the tool effector before grasp initiation based on the different factors and their interactions. The most important results suggest that for unfamiliar tools subjects fixate more on the tool's effector during the use task. Additionally, we see that under more natural conditions, subjects fixate more on the handle of the tool when it is presented on the left and in-congruent to the subjects' handedness. 
These findings lend further evidence that fixations are made in a task-oriented way which integrates prior tool knowledge and visually gathered information to plan the intended tool interaction. Furthermore, in more natural settings, fixations are also used to plan the end-state comfort of the body together with the interaction. Finally, the naturalistic setup of the present study helps us understand aspects of natural cognition that govern oculomotor behavior. 
