Seminal studies on human cognitive behavior have demonstrated that visual attention is mainly goal-directed and allocated based on the action performed. However, it is unclear how far these results generalize to cognition in more ecological and naturalistic settings. The present study investigates active inference processes revealed by eye movements during interaction with tools and the impact of action affordance of the environment. 
We presented participants with 3D tool models that were either familiar or unfamiliar, oriented congruent or incongruent to their handedness, and asked participants to interact with them by lifting or using. Importantly, we used the same experimental design in two setups. In the first experiment, participants interacted with a VR controller in a low realism environment; in the second, they performed the task with an interaction setup that allowed differentiated hand and finger movements in a high realism environment. We investigated the differences in odds of fixations and their eccentricity towards the tool parts before action initiation. Our experiment design differentiated gaze-based planning and the influence of proximal planning related to grasping the tools and the distal goal-oriented planning of acting with the tools.
The results show that participants fixate more on the tool’s effector part before action initiation for the use task and for unfamiliar tools. Furthermore, with more realistic action affordances, fixations are biased towards the handle as a function of the tool handle orientation, well before the action was executed. Secondly, the eccentricity of the fixations on the tools was influenced by the task, tool familiarity at specific periods during the viewing period in both experiments. In contrast, the spatial orientation of the tool biased the fixations towards the tool handle for the entire viewing period only in the experiment with more realistic action affordance. 
In sum, the findings from the experiments suggest that irrespective of the action affordance of the environment, fixations are made in a task-oriented way to plan the intended action well before action initiation. In the case of more realistic action affordance, anticipatory gaze is biased toward the planning the end-state comfort of grasping the tool. Taken together, our study offers a veridical and ecologically valid context to aspects of anticipatory gaze control.
