Visual attention is mainly goal-directed and allocated based on the action performed. However, it is unclear how far these results generalize to cognition in more naturalistic settings. The present study investigates active inference processes revealed by eye movements during interaction with familiar and novel tools with two levels of realism of the action affordance. In the first experiment, participants interacted with a VR controller in a low realism environment; in the second, they performed the task with an interaction setup that allowed differentiated hand and finger movements in a high realism environment. We investigated the differences in odds of fixations and their eccentricity towards the tool parts before action initiation. The results show that participants fixate more on the tool’s effector part before action initiation for the use task for unfamiliar tools. The spatial viewing bias on the tool reveals early fixations are influenced by the task and the familiarity of the tools. Later fixations are associated with the manual planning of the interaction. Our findings show that fixations are made in a task-oriented way to plan the intended action well before action initiation. With more realistic action affordances, fixations are made towards the proximal goal of optimally planning the grasp even though the perceived action on the tools is identical for both experimental setups. Taken together, proximal and distal goal-oriented planning is contextualized to the realism of action/interaction afforded by an environment.


